\documentclass{article}
% Uncomment the following line to allow the usage of graphics (.png, .jpg)
%\usepackage[pdftex]{graphicx}
% Comment the following line to NOT allow the usage of umlauts

\pagestyle{empty}
\usepackage[T2A]{fontenc}
\usepackage[utf8]{inputenc}
\usepackage[russian]{babel}
\usepackage{cmap}
\usepackage{amsthm}
\usepackage{amsmath}
\usepackage{units}
\usepackage{fancyhdr}
\usepackage{forloop}
\usepackage{amssymb}
\usepackage{url}
\usepackage{hyperref}
\usepackage{xcolor}
\usepackage[inline]{enumitem}
\usepackage{graphicx}
\usepackage{caption}
\usepackage{subcaption}
\usepackage{amscd}


\renewcommand{\thesection}{\arabic{section}}

\renewcommand{\headrulewidth}{0.4pt}
\renewcommand{\footrulewidth}{0.4pt}

\fancyfoot[L]{стр. \thepage}
\fancyfoot[R]{\today}

\fancyhead[R]{Современные приложения ДА и ФА}
%For multipage documents only!
%\fancyfoot[L]{page: \thepage}
%Uncomment this for 1-page sheets
\fancyhead[L]{Перечислительная комбинаторика}
\fancyfoot[C]{}

\pagestyle{fancy}

\renewcommand{\baselinestretch}{1.0}
\renewcommand\normalsize{\sloppypar}

\setlength{\topmargin}{-0.5in}
\setlength{\textheight}{9.1in}
\setlength{\oddsidemargin}{-0.3in}
\setlength{\evensidemargin}{-0.3in}
\setlength{\textwidth}{7in}
\setlength{\parindent}{0ex}
\setlength{\parskip}{1ex}

\newcounter{problemset}
\newcounter{totalpages}
%Here you should set the total number of pages
\setcounter{totalpages}{1}

\def \topic {Семинар 3}

\def \Z {\mathbb Z}
\def \R {\mathbb R}
\def \P {\mathbb P}
\def \C {\mathbb C}
\def \vec {\boldsymbol}

\theoremstyle{definition}
\newtheorem{lemma}{Лемма}
\newtheorem{example}{Пример}
\newtheorem*{theorem}{Теорема}
\newtheorem*{definition}{Определение}

\begin{document}

\begin{center}

\newcommand{\HRule}{\rule{\linewidth}{0.5mm}}
\HRule \\[0.2cm]
{ \Large \bfseries \topic} %\\[0.2cm]
\HRule

\end{center}

\textsc{Ключевые слова: цикловой индекс, теоремы о композиции, рекурсивные 
структуры, формула обращения Лагранжа, обобщённые деревья Кэли.}

\textbf{Напутствие.} Производящие функции интересны как \textit{инструмент} для 
решения задач, поэтому на начальных этапах мне очень хотелось бы вам предлагать 
чисто комбинаторные задачи, для которых бы требовалось найти решение, и чтобы 
производящие функции возникали в этих решениях сами по себе, чтобы они стали 
вашим удобным инструментом. Кроме получения явного ответа (вспомните формулу 
Бине для чисел Фибоначчи) иногда полезно получить какую-нибудь рекуррентную 
формулу.

В некоторых задачах я <<запрещал>> чисто комбинаторные решения, хотя в 
педагогических целях лучше условиться так: комбинаторное решение можно 
рассказывать в дополнение к решению через производящие функции, получая за это 
дополнительные очки. У меня нет цели вас подколоть и дать задачу, где бы вы 
пытались применить этот метод напрасно: у каждой задачи есть определённая цель.

Однако я хотел бы напомнить, что из общей схемы мы 
не всегда можем \textit{явно} получить вид коэффициентов производящей функции, 
иногда мы 
даже не можем явно выразить производящую функцию, а только лишь написать 
уравнение, которому она удовлетворяет. Инструменты для работы с уравнениями мы 
будем приобретать постепенно, и поверьте, они у нас будут, наша главная цель 
это асимптотика коэффициентов, причём весьма точная. Поэтому часто будут 
возникать задачи, в которых условие звучит как <<найдите производящую 
функцию>>. При этом в воздухе начинает висеть невысказанный вопрос <<а что же 
дальше?>>. Призываю набраться терпения и отложить этот вопрос на потом.

\section{Разминка}

В этом разделе будет предложено несколько задач разной степени сложности, 
потратьте около 10 минут на попытки решить каждую из них, затем посмотрите 
решение.

\textbf{Задача.} Найдите экспоненциальную производящую функцию для количества 
инволюций, то есть таких отображений \( f \colon \{ 1, 2, \ldots, n \} \to \{ 
1, 2, \ldots, n \} \), что \( f(f(x)) = x \). Затем воспользуйтесь формулой 
свёртки для экспоненциальных производящих функций и найдите формулу для 
количества инволюций (в виде суммы).

\textbf{Решение.} Заметим, что инволюция является взаимно однозначным 
отображением, то есть перестановкой, причём такой, что её цикловое разложение 
содержит только циклы длины \( 1 \) или \( 2 \). Значит, такая перестановка 
состоит из \textit{множества} циклов длины \( 1 \) и \textit{множества} циклов 
длины 2, и экспоненциальная производящая функция имеет вид
\[
	I(z) = \exp \left(
		z + z^2/2
	\right) \enspace .
\]
Формула свёртки для ЭПФ:
\[
	\left(a_0 + \dfrac{a_1}{1!}x + \dfrac{a_2}{2!}x^2 + \ldots\right)
	\left(b_0 + \dfrac{b_1}{1!}x + \dfrac{b_2}{2!}x^2 + \ldots\right)	
= \sum_{n \geq 0} x^n \sum_{k = 0}^{n}{n \choose k} a_{k} b_{n-k} \enspace ,
\]
откуда число инволюций \( I_n \) равно
\[
	I_n = \sum_{k=0}^{\lfloor n/2 \rfloor}\dfrac{n!}{(n-2k)!2^k k!} \enspace .
\]

\textbf{Задача.} Найдите экспоненциальные производящие функции для количества 
перестановок,
\begin{itemize}
	\item имеющих только циклы чётного размера,
	\item имеющих только циклы нечётного размера,
	\item имеющих чётное количество циклов,
	\item имеющих нечётное количество циклов.
\end{itemize}

\textbf{Ответ.}

\begin{itemize}
	\item \( E(z) = \exp\left(\dfrac12 \log \dfrac{1}{1-z^2}\right) = 
	\dfrac{1}{\sqrt{1 - z^2}} \) , 
	\( \qquad \bullet \) \( O(z) = \exp\left(\dfrac12 \log 
	\dfrac{1+z}{1-z}\right) = 
	\sqrt{\dfrac{1+z}{1-z}} \)
	\item \( E^{*}(z) = \mathrm{ch}\ \left(\log \dfrac{1}{1 - z}\right) = 
	\dfrac12 \dfrac{1}{1 - z} + \dfrac{1 - z}{2} \) , \( \qquad \bullet \)
	\( O^{*}(z) = \mathrm{sh}\ \left( \log \dfrac{1}{1 - z}\right) = 
	\dfrac12\dfrac{1}{1 - z} + \dfrac{z - 1}{2} \).
\end{itemize}

\section{Чтобы понять рекурсию, надо сперва понять рекурсию}

\textbf{Задача.} Давайте докажем, что числа Каталана имеют вид \( 
\dfrac{1}{n+1}{2n \choose n} \). Это доказательство настолько избитое, и старо 
как мир, и я даже затрудняюсь сказать, в каком году оно было впервые получено 
(вероятно, Лапласом или Эйлером). В любом случае, практически каждый 
первокурсник ФУПМ встречал это рассуджение, но я собираюсь его здесь повторить 
для полноты картины, и заодно напомнить, что же такое числа Каталана.

\textbf{Методическое замечание.} Внимательный читатель может заметить, что для 
вывода чисел Каталана нужно вернуться к непомеченным объектам и обыкновенным 
производящим функциям. Может возникнуть ощущение, что мы мечемся от одного типа 
объектов к другому и при этом возникает путаница. Но если разобраться, в случае 
чисел Каталана экспоненциальная и обыкновенная производящая функция оказываются 
одним и тем же объектом, и поэтому число помеченных объектов на \( n \) атомах 
отличается в \( n! \) раз.

\textbf{Решение.} Числа Каталана с точностью до сдвига индекса, имеют множество 
комбинаторных применений. Например (Википедия):
\begin{enumerate}
	\item \( C_n \) это число правильных скобочных последовательностей длины \( 
	2n \).
	\item \( C_n \) это число \textit{полных бинарных деревьев} с \( n + 1 \) 
	листьями.
	\item \( C_n \) это число плоских деревьев (то есть таких деревьев, для 
	каждой вершины которых потомки этой вершины упорядочены) c \( n + 1 \) 
	вершинами, или то же самое, плоских лесов с \( n \) вершинами.
	\item число монотонных путей в прямоугольнике \( n \times n \), которые 
	проходят под главной диагональю
	\item количество различных триангуляций правильного \( n+2 \)-угольника.
	\item число перестановок множества \( \{ 1, 2, \ldots, n \} \), в которых 
	не содержится паттерн \( 123 \).
	\item Число способов провести \( n \) диагоналей (хорд) в правильном \( 2n 
	\)-угольнике так, чтобы они не пересекались
	\item Число способов заполнить таблицу \( 2 \times n \) числами от \( 1 \) 
	до \( 2n \) так, чтобы в каждом столбце и в каждой строке числа ворастали.
\end{enumerate}
Интересующимся читателям могу посоветовать свежую книгу Стенли(2015) 
\cite{stanleycatalan}, в которых он 
перечисляет 216 комбинаторных интерпретаций чисел Каталана.

Мы рассмотрим третью интерпретацию, чтобы по возможности было меньше 
технических трудностей. Здесь мы встречаем рекурсивную конструкцию:

\textit{Плоское дерево --- это корневая вершина и последовательность плоских 
деревьев}.

\[
	T = \bullet \times \text{\textsc{seq}}(T) \enspace .
\]

Замечу, что на прошлом семинаре тоже встречалась рекурсивная конструкция, когда 
мы разбирали более сложный пример, количество down-up перестановок с нечётным 
числом вершин, там даже возникло дифференциальное уравнение, которое мы сумели 
решить. Здесь же возникает всего лишь квадратное уравнение.
\[
	T(z) = z \dfrac{1}{1 - T(z)}, \quad T^2(z) - T(z) + z = 0 \enspace ,
\]
которое имеет решение
\[
	T(z) = \dfrac{1 \pm \sqrt{1 - 4z}}{2} \enspace .
\]
Важно выбрать правильный знак перед корнем, и это будет минус, потому что \( 
T(0) = 0 \).

Про рекурсивные уравнения много говорить не надо: они вам уже могли 
поднадоесть, скажем, в курсе ТРЯП, где для изучения регулярных языков 
использовались системы производящих функций от нескольких переменных. Различие 
лишь в том, что мир комбинаторных рекурсивных спецификаций гораздо богаче. 
Попробуем накидать несколько примеров, чтобы прокачать ощущение рекурсивности:
\begin{enumerate}
	\item Последовательности, или \textit{линейные порядки} \( L \), задаются 
	рекурсивной спецификацией
	\[
		L = 1 + X \cdot L \enspace .
	\]
	(Вместо значка \( \sqcup \) я буду использовать \( + \), чтобы улучшить 
	читаемость. Значок \( X \) это класс из одного объекта, попросту атом, 
	значок \( \varepsilon \) означает класс из пустого объекта, для читаемости 
	будем писать \( 1 \).)
	
	Для линейных порядков есть ещё одна занятная рекуррентность:
	\[
		L' = L \cdot L
	\]
	Подумайте, как это доказать. (\( L' \) это результат применения оператора 
	дифференцирования)
	\item Бинарные корневые деревья \( B \) имеют спецификацию \( B = 1 + X 
	\cdot B^2 \).
	\item Коммутативные расстановки скобок \( P \): \( P = X + E_2(P) \), где 
	\( E_2 \) это множество из \( 2 \) элементов, то есть \( E_2(P) \) это 
	композиция, или класс множеств из двух обектов \( p_1, p_2 \in P \).
	\item Почему бы и не рассмотреть тернарные деревья: \( T = 1 + X \cdot T^3 
	\).
	\item \textit{Монотонные деревья}~--- это корневые деревья, построенные на 
	помеченных вершинах, у которых метки вдоль любого пути монотонны. Так как 
	оператор дифференцирования \( \dfrac{d}{ds} \) стирает атом со, скажем, 
	наименьшим номером, то 
	\[
		\dfrac{d}{ds} M = \text{\textsc{seq}}(M), \qquad
		M'(z) = \dfrac{1}{1 - M(z)}
	\]
\end{enumerate}

Так вот, для того, чтобы выписать коэффициенты функции \( T(z) \), достаточно 
воспользоваться биномиальной теоремой, то есть раскрыть выражение \( (1 - 
4z)^{1/2} = \sum_{n \geq 0}(-4z)^{n} {1/2 \choose n} \). При этом особенность 
такого биномиального коэффициента состоит в том, что его можно преобразовать:
\[
	{1/2 \choose n} = \dfrac{\tfrac12 (\tfrac12 - 1) \ldots (\tfrac12 - n + 
	1)}{n!} = \dfrac{1}{2^n} \dfrac{(2n-3)!!}{n!} (-1)^{n-1} = 
	\dfrac{(-1)^{n-1}}{2^n} \cdot
	\dfrac{1 \cdot 2 \cdot \ldots \cdot (2n-2)}{n! 2 \cdot 4 \cdot \ldots \cdot 
	(2n-2)} = \dfrac{(-1)^{n-1}}{2^{2n - 1}} \cdot \dfrac{(2n-2)!}{n! (n-1)!}
\]
Таким образом,
\[
	T(z) = \dfrac{1 - \sqrt{1 - 4z}}{2} = \sum_{n \geq 1} \dfrac{1}{n} 
	{2n-2 \choose n-1} z^n \enspace .
\]
Для полноты картины нужно заметить, что аналогичное число помеченных объектов 
равно \( n! \dfrac{1}{n}{2n-2 \choose n-1} \).




Определение циклового индекса + свойства.

Формула обращения Лагранжа

\footnotesize
\bibliographystyle{plain}
\bibliography{biblio}
    
\end{document}
