%\documentclass[a4paper,twocolumn]{article}
\documentclass[a4paper]{article}

\usepackage[english]{babel}
\usepackage[utf8]{inputenc}
\usepackage{amsmath,amsthm}
\usepackage{graphicx}
\usepackage{caption}
\usepackage{subcaption}
\usepackage{nameref}
\usepackage[colorinlistoftodos]{todonotes}
\usepackage{amssymb}
\usepackage{xcolor}
\usepackage[sc]{titlesec}
\usepackage{anysize}

\usepackage{hyperref}
\definecolor{lgreen}{rgb}{0.0, 0.48, 0.0}
\definecolor{lpurple}{rgb}{0.48, 0.0, 0.48}
\hypersetup{linktocpage,
            colorlinks=true,
            linkcolor=lgreen,
            citecolor=lpurple,
            linktoc=true}

\bibliographystyle{alpha}

%------------- TIKZ --------------------
\definecolor{bblue}{rgb}{0.2, 0.4, 0.8}
\definecolor{bgreen}{rgb}{0.2, 0.6, 0.4}
\definecolor{bred}{rgb}{0.8, 0.4, 0.2}
\definecolor{bviolet}{rgb}{0.7, 0.2, 0.7}
\definecolor{blackred}{rgb}{0.6, 0.3, 0.3}
\definecolor{blackblue}{rgb}{0.3, 0.3, 0.6}

\usepackage{tikz}
\usetikzlibrary{arrows,trees,shapes,snakes,shapes.geometric}
\tikzset{
  treenode/.style = {align=center, inner sep=0pt, text centered,
    font=\sffamily},
  arn_nn/.style = {treenode, circle, bblue, draw=bblue, 
    fill=bblue!10,
    minimum width=0.5em, minimum height=0.5em
},% arbre rouge noir, noeud rouge
  arn_n/.style = {treenode, circle, bblue, draw=bblue, 
    text width=1.5em, very thick,
    fill=bblue!10},% arbre rouge noir, noeud rouge
  arn_g/.style = {treenode, circle, bgreen, draw=bgreen, 
    text width=1.5em, very thick,
    fill=bblue!10},% arbre rouge noir, noeud rouge
  arn_r/.style = {treenode, circle, bred, draw=bred, 
    text width=1.5em, very thick,
    fill=red!10},% arbre rouge noir, noeud rouge
  arn_x/.style = {treenode, triangle, draw=black,
    minimum width=0.5em, minimum height=0.5em},% arbre rouge noir, nil
  triangle/.style = {treenode, bred, draw=bred, fill=bred!20, regular polygon, regular polygon
    sides=3, very thick, text width=1.5em }
}

%----------------- THEOREMS -------------
\newtheorem{theorem}{Theorem}
\newtheorem{remark}{Remark}
\newtheorem{lemma}{Lemma}
\newtheorem{definition}{Definition}
\newtheorem{corollary}{Corollary}
\newtheorem{problem}{Problem}


\def\SET{\mathrm{SET}}
\def\zz{ {\widehat{z}} }
\def\vec{ \mathbf }

\title{
Bivariate Generating Function for Height of Simple Trees}

\author{
Sergey Dovgal
\\
\texttt{vit.north@gmail.com}
}

\date{\today}

%\setcounter{secnumdepth}{1}

\newcommand{\comment}[1]{ {\color{bblue}\bf #1} }

\begin{document}
\maketitle


\begin{abstract}
In the current note I consider simple trees given by an equation
\(
    T(z) = z \phi(T(z))
\),
where both labelled and unlabelled trees fit well into this framework. I collect the
technical results from [Flajolet-Odlyzko] to obtain a beautiful
expression for bivariate generation function \( F(z, u) \) where \( u \) marks
tree height
\[
    \boxed{
    F(z, u) \asymp
        T(z) +
        \widetilde C (u-1) \log \dfrac{1}{\varepsilon(z) }
        +
        \widetilde C \varepsilon(z) \sum_{n \geq 1}
        \left[
            \log \dfrac{1}{1 - \tfrac{u-1}{\varepsilon n}} - \dfrac{u-1}{\varepsilon n} 
        \right]
    }
\enspace , 
\]
where the function \( \varepsilon(z) \) and the constants \( \rho \), \( \widetilde C \) are given by
\[
    \varepsilon(z) = \tau \left( \dfrac{2 \phi''(\tau)}{\phi(\tau)}
\right)^{1/2}\cdot \sqrt{1 - \tfrac z \rho} , \enspace 
    \rho = \dfrac{\tau}{\phi(\tau) } , \enspace 
    \widetilde C = 2 \dfrac{\phi'(\tau)}{\phi''(\tau)}
    , \enspace   
    \phi(\tau) = \tau \phi'(\tau) 
    \enspace .
\]
\end{abstract}

\begin{proof}
In \cite{average_height} the authors prove that
\begin{equation}
    \left.\dfrac{d}{du}F(z,u)\right|_{u=1} \underset{z = \rho}{\asymp}
    \widetilde C \log \dfrac{1}{\varepsilon(z)}
    , \enspace 
    \left. \dfrac{d^s}{du^s} F(z, u) \right|_{u=1} \underset{z = \rho}{\asymp}
    \widetilde C \Gamma(s) \zeta(s) \varepsilon(z)^{-s+1}
    \enspace .
\end{equation}
This fact is in one-to-one correspondence with the fact that the tree height is
gaussian and that factorial moments of tree
height behave like \( \mathbb E (h) = \Theta( \sqrt{n} ), \, \mathbb
E(h^{\underline j}) 
= \Theta(n^{j/2}) \), where explicit constants \( C_j \) at \( \Theta(n^j) \) provide
the multiples in generating function: for example, for \( \mathbb E h \) we have
\[
    [z^n] \log(1 - 4x)^{-1} \asymp 4^n n^{-1}, \quad
    \mathbb E h
    =
    \dfrac
        {[z^n] \left. \dfrac{d}{du} F(z, u) \right|_{u=1}}
        {[z^n] \left. F(z, u) \right|_{u=1}}
     \asymp \dfrac{4^n n^{-1}}{(\sqrt{2 \pi})^{-1} 4^n (n \sqrt{n})^{-1}} =
\sqrt{2 \pi n}
    \enspace .
\]
Their proof actually involves integration around an interesting complex contour,
so it is of great methodological value. It is also commonly known that
\( 
    T(z) \asymp \widetilde C \tau
    \left(
        \frac{2 \phi''(\tau)}
            {\phi(\tau)}
    \right)^{1/2}
    \left(
        1 - \sqrt{1 - \frac z \rho}
    \right)
\).

Following the formal Taylor expansion at \( u = 1 \), we obtain
\begin{equation}
    F(z, u) =
    \sum_{s = 0}^\infty
        \dfrac{(u-1)^s}{s!}
        \left.
            \dfrac{d^s}{du^s} F(z, u)
        \right|_{u=1}
    \underset{z = \rho}\asymp
    T(z) + (u-1)\widetilde C \log \dfrac{1}{\varepsilon(z)} 
    +
    \widetilde C
    \sum_{s \geq 2}
    \dfrac
        {\Gamma(s) \zeta(s)}{s!}
    \varepsilon(z)^{-s+1}
    \enspace ,
\end{equation}%
\vspace{-0.5cm}
\begin{multline*}
    \sum_{s \geq 2}
    \dfrac
        {\Gamma(s) \zeta(s)}{s!}
    \varepsilon(z)^{-s+1}
    =
    \left|_{\underset{\zeta(s) = \sum_{n \geq 1} \tfrac{1}{n^s}}{} }
    \varepsilon(z) \sum_{n \geq 1} \sum_{s \geq 2}
        \dfrac{( (u-1)\varepsilon^{-1} )^s}{s n^s} 
    = \right|_{t_n = (u-1) \varepsilon^{-1} n^{-1}}
    \\
    =
    \varepsilon(z) \sum_{n \geq 1} \sum_{s \geq 2} t_n^s/s
    =
    \varepsilon(z) \sum_{n \geq 1}
    \left[
        \log \dfrac{1}{1 - \tfrac{u-1}{\varepsilon n}} - \dfrac{u-1}{\varepsilon n} 
    \right]
    \enspace .
\end{multline*}
\end{proof}
\vspace{-0.7cm}
\textbf{Question.} Is there a place to study the full asymptotic Puiseux expansion in
powers of \(
\varepsilon(z) \)?

\bibliographystyle{plain}
\bibliography{biblio}

\end{document}
