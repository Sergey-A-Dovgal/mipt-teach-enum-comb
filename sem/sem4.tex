\documentclass[a5paper]{article}

% Uncomment the following line to allow the usage of graphics (.png, .jpg)
%\usepackage[pdftex]{graphicx}
% Comment the following line to NOT allow the usage of umlauts

\pagestyle{empty}
\usepackage[T2A]{fontenc}
\usepackage[utf8]{inputenc}
\usepackage[russian]{babel}
\usepackage{cmap}
\usepackage{amsthm}
\usepackage{amsmath}
\usepackage{units}
\usepackage{fancyhdr}
\usepackage{forloop}
\usepackage{amssymb}
\usepackage{url}
\usepackage{hyperref}
\usepackage{xcolor}
\usepackage[inline]{enumitem}
\usepackage{graphicx}
\usepackage{epstopdf}
\usepackage{caption}
\usepackage{subcaption}
\usepackage{amscd}

\def \topic {Семинар 4}

\renewcommand{\thesection}{\arabic{section}}

\renewcommand{\headrulewidth}{0.4pt}
\renewcommand{\footrulewidth}{0.4pt}

%\fancyfoot[RE,LO]{\thepage}
%\fancyfoot[L]{}

\fancyhead[R]{\thepage}
%For multipage documents only!
%\fancyfoot[L]{page: \thepage}
\fancyhead[L]{Перечислительная комбинаторика}
\fancyfoot[C]{\topic}

\pagestyle{fancy}

\renewcommand{\baselinestretch}{1.0}
\renewcommand\normalsize{\sloppypar}

\setlength{\topmargin}{-0.5in}
\setlength{\textheight}{6.5in}
\setlength{\oddsidemargin}{-0.3in}
\setlength{\evensidemargin}{-0.3in}
\setlength{\textwidth}{4.5in}
\setlength{\parindent}{0ex}
% \setlength{\parskip}{1ex}

\newcounter{problemset}
\newcounter{totalpages}
%Here you should set the total number of pages
\setcounter{totalpages}{1}



\def \Z {\mathbb Z}
\def \R {\mathbb R}
\def \P {\mathbb P}
\def \C {\mathbb C}
\def \vec {\boldsymbol}
\def \seq {\text{\textsc{seq}}}
\def \fprod {\; \raisebox{.3\height}{\scalebox{0.6}{$\square$}}\; }
\def \point {\raisebox{.1\height}{\scalebox{0.8}{$\Theta$}}}

\theoremstyle{definition}
\newtheorem{lemma}{Лемма}
\newtheorem{example}{Пример}
\newtheorem{corollary}{Следствие}
\newtheorem*{theorem}{Теорема}
\newtheorem*{definition}{Определение}

\usepackage{titlesec}

\makeatletter
\renewcommand{\section}{\@startsection
{section}%                   % the name
{1}%                         % the level
{\z@}%                       % the indent / 0mm
{-\baselineskip}%            % the before skip / -3.5ex \@plus -1ex \@minus 
%%-.2ex
{0.5\baselineskip}%          % the after skip / 2.3ex \@plus .2ex
{\centering\large\scshape}} % the style

\renewcommand{\subsection}{\@startsection
{subsection}%                % the name
{1}%                         % the level
{\z@}%                       % the indent / 0mm
{-\baselineskip}%            % the before skip / -3.5ex \@plus -1ex \@minus 
%%-.2ex
{0.5\baselineskip}%          % the after skip / 2.3ex \@plus .2ex
{\centering\large\scshape}} % the style
\makeatother

\begin{document}

\begin{center}

\newcommand{\HRule}{\rule{\linewidth}{0.5mm}}
\HRule \\[0.2cm]
{ \Large \bfseries \topic} %\\[0.2cm]
\HRule

\end{center}

\textsc{Ключевые слова: 
структуры с весом, доказательство теоремы о композиции для структур с весом,
полиномы Эрмита, формула Харера-Цагира
}

\section{Разминка}

\textbf{Вопрос}. Что означает последовательность букв \( F[U] \)?

\textbf{Ответ}. Из контекста символов и нашего курса должно быть
восстановлено следующее: \( F \) это некоторый класс объектов, \( U \)~--- это
множество атомов вида \( U = \{ 1, 2, \ldots, n \} \), \( F[U] \)~--- это
<<срезка>> класса \( F \) для количества атомов \( n \), иначе говоря, множество
всевозможных объектов размера \( n \) данного класса. Или множество всех
объектов, построенных на множестве атомов \( U \).

\textbf{Вопрос}. Что означает набор букв \( F[\sigma] \)?

\textbf{Ответ}. Из контекста следует, что \( \sigma \) это скорее всего
перестановка \( n \) элементов, то есть \( \sigma \in S_n \). \( F[\sigma]\)
означает <<транспорт>> вдоль перестановки \( \sigma \). Пусть множество \( F[n]
\) содержит \( \ell \) объектов. Тогда \( F[\sigma] \) задаёт некоторую
перестановку этих \( \ell \) объектов.

\textbf{Вопрос.} Что же такое тогда класс объектов?

\textbf{Ответ.} Зафиксируем некоторое число \( n \in \mathbb N \), и рассмотрим
множество \( U = \{ 1, 2, \ldots, n \} \). Класс объектов порождает множество \(
F[U] \) объектов размера \( n \) со следующим дополнительным свойством. На
множестве перестановок \( \sigma \in
S_n \) задано отображение \( \sigma \to F[\sigma] \), которое обладает
свойствами
\begin{enumerate}
 \item[a)] \( F[\sigma \circ \tau] = F[\sigma]\circ F[\tau] \).
 \item[b)] \( F[Id_n] = Id_{F[U]} \). 
\end{enumerate}

\textbf{Басня о декатегорификации стада овец (\cite{category_feynman},
\cite[Remark 6, page 11]{species}).}
Давным-давно, когда пастухи хотели понять, \textit{изоморфны} ли два стада овец,
они искали конкретный изоморфизм. Для этого было необходимо выстроить в ряд оба
стада, а затем сопоставить каждой овце из первого стада овцу из второго стада.
В один прекрасный день, пастух изобрёл декатегорификацию. Пастух понял, что овец
можно <<сосчитать>>, установив изоморфизм между множеством овец и множеством
натуральных чисел, используя бессмысленные слова типа <<один, два, три>>.
Сравнивая числа, можно было понять, изоморфны ли два стада, не предъявляя
соответсвующий изоморфизм! Короче говоря, множество \( \mathbb N \) было создано
как декатегорификация FinSet, категория конечных множеств, чьи морфизмы это
отображения конечных множеств.

Затем пастухи изобрели основные операции типа сложения, умножения,
декатегорифицировав важные теоретико-множественные операции: непересекающееся
объединение, декартово произведение, и так далее. Затем, их потомки расширили
понятие чисел, изобрели ещё более замечательные формальные операции и их
свойства: изобрели рациональные, действительные, комплексные числа, функции,
интегралы и производные, надоказывали теорем. В процессе этой деятельности
оригинальная связь с категорией конечных множеств была утеряна.

Производящая функция это выражение типа \( \sum_{n \geq 0} \dfrac{a_n}{n!} x^n
\), \( a_n \in \mathbb N \), и над такими выражениями можно проделывать разные
формальные операции. Категория \textit{классов объектов} это
категорифицированная версия кольца формальных степенных рядов. Более того,
свойства a) и b) позволяют сформулировать определение \textit{класса объектов}
на языке теории категорий: класс объектов \( F \) это \textit{функтор}
\[
    F \colon \mathbb B \to \mathbb E
\] 
из категории \( \mathbb B \) конечных множеств в категорию \( \mathbb E \)
конечных множеств \textit{и функций}.


% \section{Формула обращения Мёбиуса}
% \begin{theorem}
%     Пусть задана последовательность \( a_n \in \mathbb C \), через которую
% определена другая последовательность
% \[
%     S_n = \sum_{d | n} a_d \enspace .
% \]
% Тогда последовательность \( a_n \) можно выразить через \( S_n \) с помощью
% формулы обращения Мёбиуса:
% \[
%     a_n = \sum_{d | n} \mu (n / d) S_d \enspace ,
% \]
% где \( \mu(n) \)~--- функция Мёбиуса, заданная формулой
% \[
%     \mu(n) = \begin{cases}
%         (-1)^k, & n = p_1 p_2 p_3 \ldots p_k \enspace , \\
%         1, & n = 1 \enspace , \\
%         0, & p^2 \mid n \enspace .
% \end{cases}
% \]
% \end{theorem}
% \begin{proof}
%     Рассмотрим дзета-функцию Римана
% \[
%     \zeta(s) = \sum_{k \geq 1} \dfrac{1}{k^s} = \prod_{p \in \mathcal P} \left(
% 1 - \dfrac{1}{p^s} \right)^{-1} \enspace ,
% \]
% где \( \mathcal P \)~--- множество простых чисел. Последнее соотношение следует
% из того, что \( \dfrac{1}{1 - x} = \sum_{k \geq 0} x^k \), а любое число
% имеет каноническое разложение на простые сомножители. (Из того, что \( \zeta(1)
% = \infty \), в частности, следует, как заметил Эйлер, что простых чисел
% бесконечно много.)
% 
% Если обратить дзета-функцию, получим функцию Мёбиуса
% \[
%     M(s) = \sum_{n \geq 1} \dfrac{\mu(n)}{n^s} \enspace .
% \]
% Теперь заметим, что если последовательности \( a_n \), \( S_n \) удовлетворяют
% соотношению \( S_n = \sum_{d | n} a_d \), то можно выписать соотношение для
% \textit{производящих функций Дирихле}:
% \[
% \left(
%    \sum_{k \geq 1} \dfrac{1}{k^s} \cdot \sum_{k \geq 1} \dfrac{a_k}{k^s} = \sum_{k \geq 1}
% \dfrac{S_k}{k^s} 
% \right)
% \quad \Leftrightarrow \quad
% \left(
%     \sum_{k \geq 1} \dfrac{a_k}{k^s} = \zeta(s) \cdot \sum_{k \geq 1}
% \dfrac{S_k}{k^s} 
% \right)
% \]
% (Для того, чтобы проветить этот факт, необходимо получить формулу свёртки для
% производящих функций Дирихле). Следовательно, обращая \( \zeta(s) \), получаем:
% \[
% \left(
%     \sum_{k \geq 1} \dfrac{S_k}{k^s} = M(s) \sum_{k \geq 1} \dfrac{a_k}{k^s}
% \right)
% \quad \Leftrightarrow \quad
% \left(
%     a_n = \sum_{d | n} \mu (n / d) S_d 
% \right) \enspace .
% \]
% \end{proof}
% \begin{example}
%     Функция \( F(n) \) задана на всём натуральном ряду. Любую ли такую функцию
% \( F(n) \) можно представить в виде
% \[
%     F(n) = \sum_{d | n} f(d) \enspace ?
% \]
% \end{example}
% \begin{example}
%     Как доказать, что \( n = \sum_{d | n} \varphi(d) \)?
% \end{example}
\section{Структуры с весом}
В задачах перечислительной комбинаторики часто необходимо рассматривать разные
параметры изучаемых структур. Например, при анализе алгоритмов, хороших и
разных, нужно, например, посчитать число деревьев с заданным количичеством
вершин \( n \) и количеством листовых вершин \( k \), или даже с заданной
высотой \( h \).

    Оказывается, понятие взвешенной структуры интересно не только само по себе,
но и необходимо для доказательства теоремы о композиции.

\begin{example}
    Рассмотрим класс корневых деревьев \( \mathcal A \), и каждому корневому
дереву \( \alpha \in \mathcal A \) дополнительно сопоставим \textit{вес} \(
w(\alpha) \), равный
\[
    w(\alpha) = t^{f(\alpha)} \enspace ,
\]
где \( t \)~--- формальная переменная, \( f(\alpha) \)~--- число листьев
\(\alpha\). Это позволяет сгруппировать корневые деревья в соответствии с
дополнительным параметром <<число листьев>>. Будем говорить, что множество \(
\mathcal A[U] \) является \textit{взвешенным}, а также что переменная \( t \)
выступает в роли <<счётчика>> для числа листьев.

Взвешенное количество деревьев \( |\mathcal A[U]|_w \) определяется как сумма
весов \( w(\alpha) \):
\[
    | \mathcal A[U] |_w = \sum_{\alpha \in \mathcal A[U]} w(\alpha) = \sum_{\alpha \in
\mathcal A[U]} t^{f(\alpha)} \enspace .
\]
Простая перегруппировка слагаемых показывает, что при \( |U| = n \),
\[
    | \mathcal A[U] |_w = \sum_{k=0}^n a_{n,k} t^k \enspace ,
\]
где \( a_{n,k}\) равно количеству корневых деревьях на \( n \) вершинах, имеющих
\( k \) листьев. Подстановка \( t = 1 \) даёт эффект подсчёта каждого дерева с
весом \( 1 \), и тогда \( | \mathcal A[U] | = n^{n-1} \).
\end{example}

Обратите внимание, что понятие взвешенных классов позволяет существенно
расширить множество классов, для которых их производящие функции аналитичны в
некоторой точке. Например, рассмотрим класс всевозможных графов на \( n \) вершинах,
имеющих \( k \) рёбер. Производящая функция имеет вид
\[
    f(x, t) = \sum_{n = 0}^\infty (1 + t)^{ {n \choose 2} } x^n \enspace ,
\]
Эта функция аналитична в точке \( (x, t) = (0, -1) \).
Более того, теперь можно рассматривать классы, в которых содержится бесконечное
число объектов заданного размера \( n \), главное, чтобы \( |\mathcal A[U]|_w \)
было корректно определено, то есть коэффициент при каждом мономе вида \( \vec
t^{\vec f(\alpha)} \) был конечным.

Для взвешенных объектов также определено понятие \textit{изоморфизма}, и здесь
возникает дополнительное требование, что изоморфизм должен сохранять вес.

\begin{definition}
    Для \(w\)-взвешенного класса объектов \( F_w \) определены экспоненциальная
производящая функция, обыкновенная производящая функция и цикловой индекс:
\[
    F_w(x) = \sum_{n \geq 0} |F[n]|_w \dfrac{x^n}{n!} \enspace ,
\]
\[
    Z_{F_w} (x_1, x_2, \ldots) = \sum_{n \geq 0} \dfrac{1}{n!} \left(
        \sum_{\sigma \in S_n} |\mathrm{Fix}\; F[\sigma]|_w x_1^{\sigma_1}
x_2^{\sigma_2} \ldots
    \right)
    \enspace ,
\]
где \( |\mathrm{Fix}\; F[\sigma]|_w \) это вес множества, состоящего из
фиксированных точек под действием перестановки \( \sigma \). Другими словами,
множество \( \mathrm{Fix}\; F[\sigma] \) осталось из предыдущей конструкции, а
вес его считается теперь не как количество элементов, а как сумма весов.

Автоморфизмы сохраняют вес, следовательно отношение экивалентности определено
корректно, и 
\[
     \widetilde F_w (x) = \sum_{n \geq 0} |F[n] / \sim |_w x^n \enspace .
\]
\end{definition}

Заметим, что формулы \( F_w(x) = Z_{F_w} (x, 0, 0, \ldots) \) и \( \widetilde
F_{w} (x) = Z_{F_w} (x, x^2, x^3, \ldots) \) остаются верными. 

\begin{example}[Формула Харера-Цагира, \cite{lando}, \cite{pittel}]
\label{example:harer-zagier}
    В топологии известно утверждение, что всякая замкнутая ориентируемая
двухмерная поверхность гомеоморфна сфере, к которой приклеено конечное число
ручек. Поверхностью рода\footnote{Буква \( g \) используется в соответствии с
английским термином \textit{genus}.} \( g \) называется 
двумерная сфера с приклеенными к ней \( g \) ручками. Поверхность рода \( 0 \)
это просто сфера, поверхность рода \( 1 \) это тор. 

Замкнутые ориентируемые поверхности можно изготавливать из многоугольников,
склеивая их стороны попарно. Например, склеивание противоположных сторон
квадрата даёт тор. При склейке необходимо соблюдать ориентацию сторон, то есть
склейка квадратика, при которой получается лента Мёбиуса, запрещена. Рассмотрим производящую функцию от двух переменных
\[
    T(s, t) = 1 + 2 s \sum_{n = 0}^\infty \dfrac{T_n(t)}{(2n-1)!!} s^n \enspace
, 
\]
где \( T_n(t) = \sum_{g} \varepsilon_g(n) t^{n+1-2g} \)~--- производящая функция
для числа склеек \( \varepsilon_g(n) \) согласно роду поверхности при заданном
числе вершин (число \( n - 2g + 1 \) равно числу вершин, согласно формуле Эйлера
\( V-E+F = 2 - 2g \)).
Смысл двойного факториала в знаменателе объясняется следующим: число
всевозможных склеек в точности равно \( (2n-1)!! = 1 \cdot 3 \cdot \ldots \cdot
(2n-1) \), а значит, и число \( \dfrac{\varepsilon_g(n)}{(2n-1)!!} \) равно
вероятности получить склейку рода \( g \) при случайном склеивании, то есть
сумма всех весов, как и положено при \( t = 1 \), равна единице (Это не совсем
обычная производящая функция, так как она как бы нормирована на количество
объектов. Можете относиться к этому как к подгону). 

Для этой производящей функции существует удивительно простое и изящное выражение 
\[
    T(s, t) = \left(
        \dfrac{1+s}{1-s}
    \right)^t \enspace .
\]
\end{example}

\begin{example}[Полиномы Эрмита, {\cite[Example 16, page 89]{species}}]
    Рассмотрим взвешенный класс инволюций \( \mathrm{Inv}_w \) (то есть
перестановок \( \varphi \) со свойством \( \varphi \circ \varphi = \mathrm{Id}
\)), таких, что вес каждой перестановки равен
\[
    w(\varphi) = t^{\varphi_1} (-1)^{\varphi_2} \enspace ,
\]
где \( \varphi_1, \varphi_2 \)~--- это, как обычно, количество фиксированных
точек и количество циклов длины \( 2 \), соответственно. Выполнено комбинаторное
равенство
\[
    \mathrm{Inv}_w = \text{\textsc{set}}(X_t + (\mathcal C_2)_{-1}) \enspace ,
\]
где \( X_t \) это класс, состоящий из одного атома с весом \( t \), а \(
\mathcal C_2 \)~--- цикл размера 2. Если принять на веру тот факт, что формулы
для композиции взвешенных классов работают <<примерно похожим образом>>, то
получается ЭПФ
\[
    \mathrm{Inv}_w(x) = \exp \left(
        tx - \dfrac12 x^2
    \right) = \sum_{n \geq 0} H_n(t) \dfrac{x^n}{n!} \enspace ,
\]
где \textit{внезапно,} \( H_n(t) \) оказывается полиномом Эрмита от переменной \( t \)
степени \( n \). К вашему сведению, полиномы Эрмита являются собственными
функциями \textit{квантового гармонического осциллятора} (подробно об этом
рассказывают на физтехе в курсе квантовой механики в 6-7 семестрах
\cite[section 4.1]{mipt-quantum}), и выражаются
формулой
\[  
    H_n(t) = (-1)^n e^{t^2} \dfrac{d^n}{dt^n} e^{-t^2}
    \enspace .
\]
Ещё полиномы Эрмита являются решениями дифференциального уравнения
\[
    H_n(t)'' - t H_n(t)' + n H_n(t) = 0 \enspace ,
\]
и это можно доказать комбинаторно. Кроме того, можно показать, что эти полиномы
удовлетворяют рекуррентному соотношению
\[
    H_{n+1} (t) = t H_n(t) - n H_{n-1}(t) \enspace .
\]
\end{example}
\section{Теорема о композиции}
\begin{definition}[Композиция структур с весом]
    Пусть заданы классы объектов \( F_w \), \( G_v \), причём \( G_v \neq 0 \).
Тогда их композиция \( F_w \circ G_v \) состоит из объектов вида \( (U, \theta)
\), где \( \theta = (\pi, f, (\gamma_p)_{p \in \pi} ) \), \( \pi \)~---
разбиение множества атомов \( U \), \( f \)~--- это \( F_w \)-структура над
кусками разбиения, \( \gamma_p \)~--- семейство \( G_v \)-структур внутри куска
разбиения. При этом вес такого объекта определяется как произведение всех весов
входящих в него структур
\[
    w(\theta) = w(f)\prod_{p \in \pi} v(\gamma_p) \enspace .
\]
\end{definition}
\begin{example}
    Рассмотрим класс объектов \( \mathcal A_s \), состоящий из корневых
деревьев, где каждое дерево имеет вес \( s \). Рассмотрим также класс \(
\text{\textsc{perm}}_w \), перестановок, где каждый цикл имеет вес \( \alpha \),
то есть
\[
    \text{\textsc{perm}} = \text{\textsc{set}} (\alpha \text{\textsc{cyc}})
\enspace .
\]
Подставляя одно в другое, получаем класс эндофункций \( {\text{\textsc{end}}_v =
\text{\textsc{perm}}_w
\circ \mathcal A_s} \), где каждая эндофункция \(\psi \) имеет вес \( v(\psi) =
s^{\mathrm{rec}(\psi)} \alpha^{\mathrm{cyc}(\psi)} \), где \( \mathrm{rec}(\psi)
\) это число рекуррентных аргументов функции \( \psi \) (таких \( x \), что \( \exists k
\colon \psi^{[k]}(x) = x \)), а \( \mathrm{cyc}(\psi) \)~--- число циклов. Экспоненциальная производящая функция (уже функция от нескольких
аргументов) для данного взвешенного класса имеет вид
\[
    \text{\textsc{end}}_v(x; \alpha, s) = \left(
        \dfrac{1}{1 - s A(x)}
    \right)^\alpha \enspace .
\]
\end{example}

    Мы собираемся доказать формулу
\[
    \widetilde{F_w \circ G_v} (x) = Z_{F_w} (\widetilde G_v(x), \widetilde
G_{v^2}(x^2) ,
\widetilde G_{v^3}(x^3), \ldots ) \enspace ,
\]
а затем увидим, что формула для композиции циклового индекса \( Z_{F_{w} \circ
G_{v}} =
Z_{F_w} \circ Z_{G_v} \) следует из предыдущей формулы, используя тот факт, что
некоторые симметрические функции являются алгебраически независимыми. Обратите
внимание на то, что для первого аргумента используется весовая функция \( v \),
тогда как для второго аргумента эта функция возводится в квадрат: \( \widetilde
G_{v^2}(x^2) \), и так далее.
Заметьте, что я ещё не определил, что из себя представляет композиция цикловых
индексов \( Z_{F_w}\circ Z_{G_v} \) для взвешенных классов объектов. Это мы определим чуть позже. Посыл в
том, что для того, чтобы доказать формулу композиции для циклового индекса обычных
(не-взвешенных) классов, нам необходимо доказать формулу композиции во всей её
общности для взвешенных классов.

Напомним, что для того, чтобы получить ОПФ
некоторого класса объектов, мы рассматривали объекты с точностью до
автоморфизма. Для каждого числа атомов \( n \) было задано действие группы \(
S_n \) на всевозможных объектах размера \( n
\).

Пусть задан класс объектов \( F = F_w \). Сопоставим этому классу объектов новый класс
\( \widetilde F = \widetilde F_w \), чья экспоненциальная производщая функция является
обыкновенной производящей функцией для класса \( F \). Интуитивно говоря, структура
на \( \widetilde F \)-объекте дополнительно снабжена некоторым автоморфизмом \( F
\)-структуры этого объекта.
\begin{definition}
    Пусть задан класс объектов \( F \). Ему можно сопоставить класс \( \widetilde
F \), определённый следущим образом. Для любого множества атомов \( U \)
\[
    \widetilde F[U] = \Big\{
        (s, \alpha) \in F[U] \times \text{\textsc{perm}}[U] \mid
         \alpha \cdot s = s
    \Big\} \enspace ,
\]
где вес задаётся формулой \(w(s, \alpha) = w(s) \), а  транспорт вдоль перестановки задан формулой
\[
    \widetilde F[\sigma](\sigma, \alpha) = (\sigma \cdot s, \sigma \circ \alpha
\circ \sigma^{-1} ) \enspace .
\]
\end{definition}
С помощью леммы Бёрнсайда несложно показать, что ЭПФ для \( \widetilde F  \)
равна \( \widetilde F(x) \).
\begin{theorem}[Лемма Бёрнсайда с весом]
    Пусть задано действие конечной группы \( G \) на конечный (или суммируемый)
взвешенный класс объектов \( (Y, w) \) с весом из \( \mathbb A \). Под действием
этой группы объекты разбиваются на классы эквивалентности. Полный вес орбит
действия группы находится из формулы
\[
    |Y / G|_w = \dfrac{1}{|G|} \sum_{g \in G} |\mathrm{Fix}\; \varphi(g) |_w
\enspace ,
\]
где для каждого \( g \in G \), \( \mathrm{Fix} \; \varphi(g) \) обозначает
подмножество \( Y \) фиксированных точек перестановки \( \varphi(g) \)
\[
    \mathrm{Fix}\; \varphi(g) = \left\{
        x \in Y \mid g \cdot x = x
    \right\}
    \enspace .
\]
\end{theorem}
\begin{proof}
Обозначим \( \gamma \in Y / G \)~--- некоторый класс эквивалентности, \( x \in
\gamma \)~--- все объекты, которые лежат в данном классе эквивалентности, \(
\mathcal O(x) \)~--- орбита (или класс эквивалентности), соответствующий данному
объекту \( x \), \( \mathcal G_x \)~--- множество элементов группы \( G \),
которые оставляют элемент \( x \) на месте.

Воспользуемся известным фактом: \( \forall x \; |\mathcal O(x) | \cdot |\mathcal
G_x| = |G| \).
    \begin{multline*}
    |Y / G|_w = \sum_{\gamma \in Y / G} w(\gamma) = 
    \sum_{\gamma \in Y / G} \dfrac{1}{|\gamma|} \sum_{x \in \gamma} w(x)
    = \sum_{x \in Y} \dfrac{w(x)}{|\mathcal O(x)|}= 
    \\
    \sum_{x \in Y} \dfrac{|\mathcal G_x|}{|G|} w(x) 
    =
    \dfrac{1}{|G|} \sum_{x \in Y} \sum_{g \in G} w(x) \mathbb I(g \cdot x = x)
    =
    \dfrac{1}{|G|} \sum_{g \in G} |\mathrm{Fix} \; \varphi(g)|_w \enspace .
    \end{multline*}
\end{proof}
\begin{definition}
Пусть задано действие конечной группы \( G \) на конечном множестве \( Y \).
\textit{Цикловой многочлен} \( P_{G} (y_1, y_2, y_3, \ldots) \) это многочлен от
переменных \(y_1, y_2, y_3, \ldots \), определённый как
\[
    P_{G} (y_1, y_2, y_3, \ldots) = \dfrac{1}{|G|} 
    \sum_{g \in G} y_1^{\varphi(g)_1} y_2^{\varphi(g)_2} y_3^{\varphi(g)_3}
\ldots \enspace ,
\]
где \( \varphi \colon G \to \text{\textsc{perm}}[Y] \) это гомоморфизм,
сопоставленный действию \( G \) на \( Y \).
\end{definition}
\begin{example}[Группа вращений куба]
    В кубе существует 4 типа поворотов: 6 штук на 180 градусов вдоль оси,
соединяющей середины противоположных рёбер, 3 оси поворотов, соединящих
центры противоположных граней, 4 оси, соединяющие противоположные вершины куба,
повороты на \( \pm 120^\circ \), тождественный поворот. Эти вращения вносят
вклад, соответственно,
\[
    (6y_2^4) + (6y_4^2 + 3y_2^4) + (8y_1^2 y_3^2) + (y_1^8)
\]
Значит, цикловой многочлен равен
\[
    P_{G} (y_1, y_2, y_3, y_4) = \dfrac{1}{24}
    (y_1^8 + 8 y_1^2 y_3^2 + 9 y_2^4 + 6 y_4^2) \enspace .
\]
\end{example}
% ------------------

Как вы видите, цикловой многочлен (многочлен, а не бесконечный степенной ряд),
имеет связь с группами, а объект, который мы ввели ранее, \textit{цикловой
индекс}, отражает действие группы автоморфизмов на классе объектов. Следующим
шагом мы установим связь между цикловым индексом и цикловым многочленом. 

\begin{theorem}
    Пусть \( F = F_w \)~--- класс взвешенных объектов. Тогда для его циклового
индекса выполнено соотношение
\[
    Z_{F}(x_1, x_2, x_3, \ldots) = \sum_{t \in T(F)} w(t) P_{\mathcal G_t} (x_1,
x_2, \ldots) \enspace ,
\]
где \( t \) пробегает по всем классам эквивалентности \( F\)-объектов (все
объекты в одном классе эквивалентности изоморфны), \( \mathcal G_t\)~--- стабилизатор \( t \)
внутри группы перестановок \( S_n \), \(P_{\mathcal G_t} \)~--- цикловой
многочлен указанного стабилизатора.
\end{theorem}
\begin{proof}
Так как построенный нами класс \( \widetilde F \) содержит информацию об
автоморфизмах объектов, то согласно определению циклового индекса,
\begin{multline*}
    Z_F(x_1, x_2, x_3, \ldots)  = \sum_{n \geq 0} \dfrac{1}{n!} \sum_{\sigma \in
S_n} |\mathrm{Fix}\; F[\sigma]|_w x_1^{\sigma_1} x_2^{\sigma_2} \ldots \\
 = \sum_{n \geq 0} \dfrac{1}{n!} \sum_{(s, \sigma) \in \widetilde F[n] }
w(s) x_1^{\sigma_1} x_2^{\sigma_2} \ldots
= \sum_{n \geq 0} \dfrac{1}{n!} \sum_{s \in F[n] } w(s) \sum_{\sigma \in
\mathcal G_s} x_1^{\sigma_1} x_2^{\sigma_2} \ldots
\end{multline*}
Обозначим \( \mathcal O(t) \) орбиту объекта \( t \). Из теории групп известно,
что \( |\mathcal O(t) | = n! / |\mathcal G_t| \).
Группируя объекты по их классам эквивалентности, получаем 
\begin{multline*}
    = \sum_{n \geq 0} \sum_{t \in T(F_n)} \dfrac{| \mathcal O(t)|}{n!} w(t)
    \sum_{\sigma \in \mathcal G_t} x_1^{\sigma_1} x_2^{\sigma_2} \ldots =
    \sum_{n \geq 0} \sum_{t \in T(F_n)} \dfrac{w(t)}{|\mathcal G_t|}
\sum_{\sigma \in \mathcal G_t} x_1^{\sigma_1} x_2^{\sigma_2} \ldots\\
= \sum_{t \in T(F)} w(t) P_{\mathcal G_t} (x_1, x_2, \ldots)
\enspace 
\end{multline*}
\end{proof}

\begin{corollary}
    Для взвешенного класса объектов \( F = F_w\) выполнено
\[
    \widetilde F_w(x) = Z_{F_w} (x, x^2, x^3, \ldots) \enspace .
\]
\end{corollary}
\begin{proof}
\begin{multline*}
    \widetilde F_w(x) = \sum_{n \geq 0} |T(F_n)|_w x^n = 
    \sum_{n \geq 0} \sum_{t \in T(F_n)} w(t) x^n = 
    \sum_{n \geq 0} \sum_{t \in T(F_n)} \dfrac{w(t)}{| \mathcal G_t|}
\sum_{\sigma \in \mathcal G_t} x^n
\\= Z_F(x, x^2, x^3, \ldots)\enspace .
\end{multline*}
\end{proof}

\begin{theorem}
    ОПФ для композиции взвешенных классов \( F_w \) и \( G_v \) имеет вид:
\[
    \widetilde{F_w \circ G_v}(x) = Z_{F_w}
    (\widetilde G_v(x), \widetilde G_{v^2}(x^2), \ldots)
    \enspace.
\] 
\end{theorem}
\begin{proof}
    Начнём с того, что проанализируем типичную \( \widetilde{F \circ G}
\)-структуру на множестве атомов \( U \). Напомним, что такая структура является
парой \( (s, \sigma) \), где \( \sigma \in \text{\textsc{perm}}[U] \), и \( s \)
является \( F \circ G \)-структурой на \( U \), которая остаётся неподвижной
относительно перестановки\( \sigma \). При этом \( s = ( \pi, f, g ) \), где
\begin{enumerate}
\item \( \pi \) является разбиением \( U \),
\item \( f \) является \( F \)-структурой над \( \pi \),
\item \( g= (g_p)_{p \in \pi} \)~--- набор \( G \)-структур, где каждая \( g_p
\) является \( G\)-структурой над соответствующим подмножеством атомов.
\end{enumerate}
Перестановка \( \sigma \) должна индуцировать перестановку классов \(
\sigma_\pi\), \( \sigma
\cdot \pi = \pi \), каждый блок \( p \in \pi \) целиком переходит в какой-то
другой блок \( \sigma(\pi) \). Кроме того, внутри каждого класса мы имеем
ограничение перестановки \( \sigma \) на этот класс \( \sigma |_p \colon g_p
\to g_{\sigma(p)} \). При этом вес любого объекта из \( \widetilde{F \circ G} \)
равен
\[
    w(s, \sigma) = w(f) \prod_{p \in \pi} v(g_p) \enspace .
\]
Пусть \( N = (n_1, n_2, n_3, \ldots) \)~--- некоторый цикловой тип некоторой
перестановки. Обозначим \( (\widetilde{F \circ G})_N \)~--- подкласс класса \(
\widetilde{F \circ G} \), для которых индуцированная перестановка \( \sigma_\pi
\) имеет цикловой тип \( N \).
Ясно, что
\[
    \widetilde{F \circ G} = \sum_N (\widetilde{F \circ G})_N \enspace ,
\]
где сумма берётся по всевозможным наборам последовательностей \( (n_1, n_2,
\ldots ) \).

Назовём \textit{сплетением \( G\)-структур} структуру типа \(
\widetilde{\text{\textsc{set}} \circ G} \), чья индуцированная перестановка \(
\sigma_\pi \) состоит лишь из одного цикла, и обозначим \( K_m (G) \) класс
таких сплетений, для которых циклы имеют длину \( m \).

\textbf{Упражнение.} Покажите, что производящая функция для композиции \(
K_m(G_v)(x) \) задаётся формулой
\[
    K_m(G_v)(x) = \dfrac{\widetilde{G_{v^m}}(x^m)}{m}
    \enspace .
\]
Заметим, что при этом любая \( \widetilde{F \circ G} \)-структура целиком
задаётся набором сплетений соответствующих \( G \)-структур, то есть
\[
    (\widetilde{F \circ G})_N (x) = |\mathrm{Fix} \; F[N] |_w
    \dfrac{(K_1(G)(x))^{n_1}}{n_1!}
    \dfrac{(K_2(G)(x))^{n_2}}{n_2!}\ldots
\]
Осталось заметить, что
\begin{multline*}
    \widetilde{F_w \circ G_v}(x) = \sum_{|N| < \infty} (\widetilde{F_w \circ
G_v})_N(x)\\ = \sum_{n_1 + 2 n_2 + \ldots < \infty} | \mathrm{Fix}\; F[n_1, n_2,
\ldots] |_w 
    \dfrac{\left(\widetilde{G_v}(x)\right)^{n_1}}{1^{n_1} n_1!}
    \dfrac{\left(\widetilde{G_v}(x)\right)^{n_2}}{2^{n_2} n_2!} \ldots
 \\= 
Z_{F_w} (\widetilde{G_v}(x), \widetilde{G_{v^2}} (x^2), \ldots) \enspace ,
\end{multline*}
что и требовалось доказать. 
\end{proof}


% ------------------

\begin{theorem}
    Пусть задан класс объектов \( F = F_w \) с весом из \( \mathbb A \). Тогда цикловой индекс \( Z_{F_w} \)~--- это
единственный формальный степенной ряд \( f(x_1, x_2, x_3, \ldots) \), такой, что
для любого класса объектов \( G \) выполнено
\[
    \widetilde{F_w \circ G_v}(x) = f(\widetilde G_v(x), \widetilde G_{v^2}(x^2), \widetilde
G_{v^3}(x^3), \ldots) \enspace .
\] 
\end{theorem}
\begin{proof}
    Мы уже выяснили, что цикловой индекс \( f = Z_{F_w} \) удовлетворяет
указанному соотношению. Докажем единственность. Выберем переменные \( t_1, t_2,
t_3, \ldots \), которые не принадлежат кольцу \( \mathbb A \), а затем
рассмотрим класс объектов
\[
    G_v = X_{t_1} + X_{t_2} + X_{t_3} + \ldots
\]
Для него выполнено
\[
    \widetilde{G_{v^k}}(x^k) = (t_1^k + t_2^k + t_3^k + \ldots) x^k = S_k x^k
\enspace ,
\]
где \( S_k \)~--- симметрическая сумма степеней \( k \).
Известный способ показать, что две функции совпадают на всевозможных наборах
аргументов~--- это рассмотреть их разность, как функцию от того же набора
аргументов, и доказать, что она всегда равна нулю. Наш путь~--- рассмотреть не
все наборы аргументов, а только те, которые мы указали выше в качестве примера с
симметричными функциями. Следовательно, для того, чтобы придерживаться такого
плана, достаточно доказать,
что для функции \( f \in \mathbb A [ [ x ] ] \) выполнено
\[
    \forall \vec t, x \ 
    f(S_1 x, S_2 x^2, S_3 x^3, \ldots ) = 0 \quad 
    \Rightarrow \quad
    \forall \vec x\ 
    f(x_1, x_2, x_3, \ldots) = 0 \enspace .
\]
Если \( f(x_1, x_2, x_3, \ldots) = \sum_{n_1, n_2, \ldots} a_{n_1,n_2, \ldots}
x_1^{n_1} x_2^{n_2} \ldots \), то положим
\[
    f_n(x_1, x_2, x_3, \ldots) = \sum_{n_1 + 2n_2 + \ldots = n}
    a_{n_1, n_2, \ldots} x_1^{n_1} x_2^{n_2} \ldots \enspace .
\] 
Тогда \( f_n \) является полиномом от \( x_i \) и 
\begin{eqnarray*}
    f(S_1 x, S_2 x^2, S_3 x^3, \ldots) = 0 & \Rightarrow & 
    \sum_{n \geq 0} f_n(S_1, S_2, S_3, \ldots) x^n = 0\\
    & \Rightarrow & f_n(S_1, S_2, S_3, \ldots) = 0 , \\
    & \Rightarrow & f_n(x_1, x_2, x_3, \ldots) = 0, \\
    & \Rightarrow & f(x_1, x_2, \ldots) = 0 \enspace .
\end{eqnarray*}
в силу того, что симметрические функции \( S_k = S_k(t_1, t_2, \ldots) \)
являются алгебраически независимыми\footnote{Мне кажется, это легко доказать с
помощью определителя Вандермонда, хоть и число переменных бесконечно. Если бы
существовала конечная линейная комбинация, это бы привело к конечномерному
противоречию.} над \( \mathbb A \). 
\end{proof}

Наконец, мы можем доказать финальный результат.
\begin{theorem}
    Для любых двух взвешенных классов \( F_w, G_v \), для которых \( G(0) = 0
\), верна формула для композиции циклового индекса:
\begin{multline*}
    Z_{F_w \circ G_v} 
= Z_{F_w}( Z_{G_v}(x_1, x_2,\ldots  ),
           Z_{G_{v^2}}(x_2, x_4, \ldots),
           Z_{G_{v^3}}(x_3, x_6, \ldots), \ldots)\enspace .
\end{multline*}
\end{theorem}
\begin{proof}
    Воспользуемся результатом предыдущей теоремы. Для этого рассмотрим
произвольльный взвешенный класс \( H_u \) и докажем, что
\[
    \widetilde{ (F_w \circ G_v) \circ H_u}(x) = (Z_{F_w} \circ Z_{G_v})
    \left(
        \widetilde{H_u}(x), \widetilde{H_{u^2}}(x^2), \ldots
    \right)
\]
Этого будет достаточно для того, чтобы установить формулу для композиции
цикловых индексов. Нагло воспользуемся тем, что композиция является
ассоциативной (проверьте!), откуда следует, что
\begin{multline*}
    \widetilde{ (F_w \circ G_v) \circ H_u}(x) =
    \widetilde{ F_w \circ (G_v \circ H_u)}(x) \\=
    Z_{F_w} \left(
        \widetilde{G_v \circ H_u}(x),
        \widetilde{G_{v^2} \circ H_{u^2}}(x^2),
        \ldots
    \right)  \\
= 
    Z_{F_w} \left(
        Z_{G_v}(\widetilde{H_u}(x), \widetilde{H_{u^2}}(x^2), \ldots),
        Z_{G_{v^2}}(\widetilde{H_{u^2}}(x^2), \widetilde{H_{u^4}}(x^4), \ldots),
        \ldots
    \right)  \\
= (Z_{F_w} \circ Z_{G_v}) (\widetilde{H_u}(x), \widetilde{H_{u^2}}(x^2), \ldots)
\enspace .
\end{multline*}
\end{proof}

\section{Задачи}

\begin{enumerate}
    \item(2 очка) Выпишите производящую функцию для числа склеек \( 2n \)-угольника,
дающих тор. Можно воспользоваться книгой Ландо \cite{lando}, там нет решения, но
есть целая глава, посвящённая этой теме.
    \item(1 очко) Докажите, что функция \( T_n(t) \) из примера
\ref{example:harer-zagier}
является чётной при нечётном \( n \) и нечётной при чётном \( n \).
\end{enumerate}

\footnotesize
\bibliographystyle{plain}
\bibliography{biblio}
    
\end{document}
